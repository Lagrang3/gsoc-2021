% rubber: bibtex.path /home/lagrange/Documents/UNDOCUMENTS/gsoc/2021/proposal
\documentclass[11pt,a4paper]{article}

\usepackage{amsmath}
\usepackage{amssymb}
\usepackage{amsthm}
\usepackage[a4paper,margin=1in]{geometry}
\usepackage{graphicx}
\usepackage{subcaption}
\usepackage{hyperref}

\hypersetup{
    colorlinks=false,
    hidelinks=true
}
\bibliographystyle{plain} % FIXME: styles are not found

\newcommand{\fft}{\textsc{fft}}
\newcommand{\dft}{\textsc{dft}}
\newcommand{\gsoc}{\textsc{gsoc}}
\newcommand{\ntt}{\textsc{ntt}}
\newcommand{\mpi}{\textsc{mpi}}
\newcommand{\raii}{\textsc{raii}}
\newcommand{\api}{\textsc{api}}
\newcommand{\boost}{\textit{Boost}}
\newcommand{\boostmath}{\textit{Boost.Math}}
\newcommand{\boostmpi}{\textit{Boost.MPI}}
\newcommand{\fftw}{\textit{FFTW}}
\newcommand{\cufft}{\textit{CUFFT}}
\newcommand{\gcc}{\textit{GCC}}
\newcommand{\gnu}{\textit{GNU}}
\newcommand{\gsl}{\textit{GSL}}
\newcommand{\alglib}{\textit{ALGLIB}}

\newcommand{\Ring}{\mathcal{R}}
\newcommand{\Order}{\mathcal{O}}
\newcommand{\Complex}{\mathbb{C}}
\newcommand{\Real}{\mathbb{R}}
\newcommand{\Integer}{\mathbb{Z}}
\newcommand{\xroot}[1]{$#1^{\mathrm{th}}$-root}

\newtheorem{example}{Example}
\newtheorem{definition}{Definition}

\title{\gsoc\ 2021\\\boostmath: \fft\ Utilities}
\author{Eduardo Quintana Miranda}


\begin{document}


\maketitle
\begin{abstract}
We propose the implementation of Fast Fourier Transform (FFT) capabilities
inside the Boost.Math library. This project answers the need for a C++ library
of this sort, and the lack of FFT within Boost.Math.
Our goal is mainly to provide FFT template routines that support types that
satisfy the Ring Axioms, and not just complex numbers. 
\end{abstract}

\tableofcontents
\input intro
\input state
\input proposal
\input schedule
    
\bibliography{biblio}

\appendix
\input app_ring
\input info

\end{document}

Elements of a Quality Proposal

Most organizations have their own proposal guidelines or templates. You should
be extraordinarily careful to conform to these. Most organizations have many,
many proposals to review. Failure to follow simple instructions is highly likely
to land you at the bottom of the heap.

There are certain elements of the proposal that should apply to every
organization. Proper attention to these elements will greatly improve your
chances of a successful proposal.

Name and Contact Information

Putting your full name on the proposal is not enough. Provide full contact
information, including email addresses, websites, IRC nick, and telephone
number.

Title

Your title should be short, clear and interesting. The job of the title is to
convince the reviewer to read your synopsis.

Synopsis

If the format allows, start your proposal with a short summary, designed to
convince the reviewer to read the rest of the proposal.

Benefits to Community

Don’t forget to make your case for a benefit to the organization, not just to
yourself. Why would Google and your organization be proud to sponsor this work?
How would open source or society as a whole benefit? What cool things would be
demonstrated?

Deliverables

Include a brief, clear work breakdown structure with milestones and deadlines.
Make sure to label deliverables as optional or required. You may want your plan
to start by producing some kind of white paper, or planning the project in
traditional Software Engineering style. It’s OK to include thinking time
(“investigation”) in your work schedule. Deliverables should include
investigation, coding and documentation.

Related Work

You should understand and communicate other people’s work that may be related to
your own. Do your research, and make sure you understand how the project you are
proposing fits into the target organization. Be sure to explain how the proposed
work is different from similar related work.

Biographical Information

Keep your personal info brief. Be sure to communicate personal experiences and
skills that might be relevant to the project. Summarize your education, work,
and open source experience. List your skills and give evidence of your
qualifications. Convince your organization that you can do the work. Any
published work, successful open source projects and the like should definitely
be mentioned.

Follow the Rules

Most organizations accept only plain text applications. Most organizations have
a required application format. Many organizations have application length
limits. In general, organizations will throw out your proposal if you fail to
conform to these guidelines. If you feel you must have graphical or interactive
content associated with your application, put just this content (not a copy of
your proposal) on the web and provide an easy-to-type URL. Do not expect
reviewers to follow this link.


