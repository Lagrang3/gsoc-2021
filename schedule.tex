\section{Timeline}

    \subsection*{Before June 7}
    Familiarize myself with \boostmath\ library, its coding standards and
    requirements for new utilities.
    Familiarize myself with \fftw\ library, its API and inner workings.
    Do some research about the state of the art algorithms for computing \fft.
    
    \subsection*{June 7 -- June 13 (Official Coding Time begins)}
    Define a template API for the \fft\ library.
    Prepare a set of tests for correctedness.
    Prepare a set of benchmarks to test performance.
    Implement, test and document an abstract
    \fft\ using multi-radix Good-Thomas \cite{FFTW05}
    algorithm (most general method). 
     
    
    \subsection*{June 14 -- June 20}
    Write a routine for computing convolutions using \fft.
    Implement, test and document a
    complex-to-complex \fft\ using Rader's algorithm \cite{rader68}.
    
    \subsection*{June 21 -- June 27}
    Implement Winograd's \fft\ \cite{winograd78} as a generalization of Rader's
    and check if there is a gain in performance for certain \fft\ sizes.
    Implement, test and document a
    complex-to-complex \fft\ using Bluestein's algorithm \cite{bluestein70} and check if there is a
    gain in performance for certain \fft\ sizes.
    
    
    \subsection*{June 28 -- July 04}
    Implement, test and document
    real-to-complex and complex-to-real 
    \fft's algorithms discussed in \cite{FFTW05,soresen87}.
    
    \subsection*{July 05 -- July 11}
    Buffer time before the first evaluation date
    to solve any pending issues.
    
    \subsection*{July 12 -- July 18}
    Benchmark the \fft\ utilities produced in the preceding weeks against \fftw,
    Alglib and GSL libraries\footnote{\url{https://www.alglib.net/}}.
    Explore other algorithms and techniques, for instance \fftw\
    codelets.
    
    \subsection*{July 19 -- July 25}
    Implement a wrapper to \fftw\ libraries. 
    Specialize the template \fft\ to call FFTW when the required type of
    transform can be handled by FFTW's API,
    eg. single/double precission complex to complex, or single/double
    precission real to complex.
    
    \subsection*{July 26 -- August 1}
    Document the work done. Produce some examples for the most common \fft\
    cases as well as some cases when FFTW cannot be used and templated
    Cooley-Tukey is called instead.
    Implement, test and document, $D$-dimensional \fft\ using the 1-dimensional
    functionality produced in the previous weeks.
    
    \subsection*{August 2 -- August 8}
    Implement, test and document, some multi-threading capabilities.
    
    \subsection*{August 9 -- August 15}
    Buffer time before the last evaluation date
    to solve any pending issues.
   
%    Proposal Timeline
% 
% Before April 20:
% 
%     To familiarize myself completely with Mailman2’s functionality and
%     architecture.  Study of the customized files of Systers Mailman available
%     in the Launchpad Baazar version control.  To familiarize myself with
%     Storm(ORM that we will be using)
% 
% April 20 – May 23 (Before the official coding time):
% 
%     To do self coding with Storm to improve my further understanding and ease
%     of use with this ORM and database(PostgreSQL) During this period I will
%     remain in constant touch with my mentor and the Mailman community. I will
%     remain active on IRC and Mailing lists to discuss and finalize on the
%     modifications (if any) that needs to be on existing schemas and design of
%     new schemas (if needed to fit cleanly with Mailman3’s Architecture) Thus
%     with the help of my mentor I will become absolutely clear about my future
%     goals,the final database implementations that need to be done as well as
%     the approach that I will follow to map the schemas to the Object Oriented
%     Paradigm.
% 
% May 23 – June 18 (Official coding period starts):
% 
%     Define all the required Relations(Tables) in my local database using
%     STORM.  Define all the corresponding Python Classes and Objects that will
%     store,modify and retrieve data in database.  Define all the interactions
%     that Systers perform with their database (virtualize or stimulate all
%     interactions) in STORM that will deal with my local database.
% 
% This will help in testing of the proper working of the entire basic code
% changes that we will later on incorporate in Systers Source code.
% 
% June 18 – July 5:
% 
%     Bringing about the decided changes in the Relational Schemas of Systers
%     database.  Replacing parts of the above code in their respective places in
%     the Systers source code. (This should not take much time as most of the
%     functionality has been implemented in the previous step).  Testing the
%     overall working of each and every module of the modified source code with
%     the help of Python Test Suites.
% 
% JULY 6th MID TERM EVALUATION
% 
% July 6 – July 15:
% 
%     Making further changes in the code to improve the Functionality, Exception
%     handling, Bug Removal.
% 
% July 15 – July 25:
% 
%     To be in constant touch with the Mailman3’s developers and to let them
%     know about our progress.  Most of the time will be consumed for rigorous
%     testing and bug fixes.
% 
% July 25 – July 31:
% 
%     For Documentation
% 
% A buffer of two weeks has been kept for any unpredictable delay.


% Timeline
% 
% This week-by-week timeline provides a rough guideline of how the project will
% be done.
% 
% 3 – 16 May
% 
% Familiarize with the code and the community, the version control system, the
% documentation and test system used, and the new Reactome version.
% 
% 17 – 30 May
% 
% Write the Reactome XML layout schema and the command line Reactome XML to GPML
% converter, keeping in mind that the internals are to be used subsequently as a
% library.
% 
% 31 May – 6 June
% 
% Test and document existing code more thoroughly.
% 
% 7 – 20 June
% 
% Determine algorithms used to convert GPML graphical representations to
% Reactome XML. Then, write the command line GPML to Reactome converter, keeping
% in mind that the internals are to be used subsequently as a library.
% 
% 21 – 27 June
% 
% Test and document the GPML to Reactome XML converter and the heuristic
% algorithm more thoroughly.
% 
% 28 June – 11 July
% 
% Ensure that round-trip conversion works flawlessly (i.e. no data is lost when
% converting GPML to Reactome XML to GPML again, and vice versa). Also test and
% document round-trip conversions.
% 
% 12 – 25 July
% 
% Integrate the converters to WikiPathways. A system that periodically check for
% updates on both WikiPathways and Reactome and update the websites accordingly
% is written.
% 
% 26 July – 1 August
% 
% Test and document the periodic push/pull mechanism more thoroughly.
% 
% 2 – 16 August
% 
% Further refine tests and documentation for the whole project.
