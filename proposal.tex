\section{Proposal}

This project proposes the implementation, with C++ templates, of well-known
\fft\ algorithms to solve the \dft\ problem as presented in the Definition
\ref{def:dft}, for any data type that satisfies the ring axioms. 
These routines will be integrated within \boostmath\ library.

\subsection{Deliverables}

This project proposes to deliver at the end of the \gsoc-2021 a C++ library for
computing \fft{}s, with the following components:
\begin{itemize}
    \item template routines for computing \dft,
    \item template routines for computing convolutions,
    \item optimized specialization for complex to complex \dft,
    \item optimized specialization for real to complex \dft,
    \item support for $D$-dimensional \dft{}s,
    \item unit tests and benchmarks,
    \item documentation with examples,
    \item (if time permits) wrappers for some \fftw\ routines,
    \item (if time permits) with multithreaded capabilities.
\end{itemize}

\subsection{Future outlook}
If this project is successful we will consider for next year 
implementing parallel distributed memory
$D$-dimensional \dft\ 
using a scalable $(D-1)$-dimensional domain decomposition,
similar to that proposed by \cite{pippig_13}, built on top of the \boostmpi\ library
and the 1-dimensional \fft\ utilities of \boostmath\ to be developed this year.
