\section{Application Questionnaire}

\subsection*{Personal details}

\begin{tabular}{ll} 
    Name:& Eduardo Quintana Miranda\\ 
    Affiliation:& University of Trieste, Astronomical Observatory of Trieste\\
    Course:& Astrophysics \\
    Degree Program:& PhD \\ 
    E-mail:& \url{eduardo.quintana@pm.me} \\ 
    Github:& \url{github.com/Lagrang3} 
\end{tabular}

\subsection*{Availability}

\subsubsection*{Q. How much time do you plan to spend on your GSoC?} 
I think that
25 hours a week for the whole duration of the event will fit for the objectives
set forth on this project.

\subsubsection*{Q. What are your intended start and end dates?} 
I intend to follow
the official timeline, ie. to work continuously from June 7th until August 16th.
However, it is very likely that I will be working on the project even before
that date on discrete time slots starting from April the 1st.

\subsubsection*{Q. What other factors affect your availability?} 
The progress on
the PhD project in which I am involved is an important factor that could limit
my availability.

\subsection*{Background Information}

\subsubsection*{Q. Please summarize your educational background.}

\begin{itemize} 
    \item 2008--2013. BSc. Nuclear Physics, University of Havana,
    \item 2015--2018. MSc. Theoretical Physics, University of Trieste, 
    \item
    2018--2019. Master in High Performance Computing, International School of
    Advanced Studies (SISSA), Trieste, 
    \item 2019--present. PhD in astrophysics,
    University of Trieste.  
\end{itemize}

\subsubsection*{Q. Please summarize your programming background.} 
My main
programming language is C++.

I started programming back in 2009 during my 2nd year of BSc.  The main drivers
were physics simulations and programming competitions.  From 2009 until 2013
I've participated intensively in those kind of competitions, 
the most important were the
ACM-ICPC\footnote{\url{icpc.global}} competitions held every year, my team
managed to classify many times to the regional phase, which in our geographical
context where named \emph{Caribbean Finals of the ICPC}.  Until this date I
participate once in a while on codeforces\footnote{\url{codeforces.com}} rounds.
I have a fair knowledge of a variety of classical algorithms somewhat seasoned
by my participation on programming competitions.

I've acquired some professional programming skills during the Master in High
Performance Computing. Advanced C++, python, bash, git, and parallel programming
in OpenMP, MPI and Cuda where among the topics taught in that curriculum.

I've read Stroustrup's \emph{The C++ Programming Language} 2nd Ed.  and I am
currently going through Josutti's \emph{C++17 The Complete Guide}.

I use C++ on a daily basis for my research. 


\subsubsection*{Q. Please tell us a little about your programming interests. Please
tell us why you are interested in contributing to the \boost\ C++ Libraries.} 
For
my PhD research I am developing a library for a very specific task, which is the
Particle-Mesh (PM) component for cosmological N-body simulations that will
include the effects of general relativity on the formation of large scale
structures---ie. clusters of galaxies, voids, filaments, etc. I use \boost\
Libraries for this work, in particular the MPI and Unit Test Framework. That
means some of my interest comes from professional needs.  But besides that, I am
interested in programming for passion.


\subsubsection*{Q. What is your interest in the project you are proposing?} 
Most of
the PM code, in which I am working on, involves performing parallel multiprocess
\fft{}s on a discrete representations of fields---eg. density---in a cubic box. At
its current state, the code has two major programming issues: 
\begin{enumerate}
    \item it uses \fftw\ to perform 1-dimensional \fft{}s calling C style routines, 
    consequently the internal datatype for the field's Fourier transform is 
    \verb|fftw_complex|, an alias for \verb|double[2]|, which lacks of the
    algebraic semantics that \verb|std::complex<double>| has;
    \item while the parallelization of the 3-dimensional transform is done by
    handcrafted
    routines because \fftw's algorithm does not scale well with the number of MPI
    processes.  
\end{enumerate} 
The project I am proposing would solve the first
issue, and my feeling is that it won't take long until we can implement scalable
$D$-dimensional \fft{}s within \boostmpi\ library using the resulting code from this
effort.

\subsubsection*{Q. Have you done any previous work in this area before or on similar
projects?}
Yes. One of the courses taken at the Master in High Performance Computing dealt
with Fast Fourier Transforms. And for the final exam we had to implement a
parallel 3-dimensional \fft\ and compare its performance against
\fftw\footnote{\url{https://github.com/Lagrang3/P2.16_Parallel_FFT/tree/master/solution}}.

\subsubsection*{Q. What are your plans beyond this Summer of Code time frame for your
proposed work?} 
If this project's idea is well received, I will continue working on it beyond
this year's \gsoc. I would like to have these \fft\ capabilities in the
upcomming \boost\ releases and become a regular maintainer of the code.
Like I said before, I have plans to continue working on it at least to the point
of having distributed memory \fft\ implemented.

\subsubsection*{Q. Please rate, from 0 to 5 (0 being no experience, 5 being expert),
your knowledge of the following languages, technologies, or tools:}
\begin{itemize} 
    \item C++ 98/03 (traditional C++): 4 
    \item C++ 11/14 (modern C++): 4 
    \item C++ Standard Library: 4 
    \item Boost C++ Libraries: 3 
    \item Git: 3
\end{itemize}

\subsubsection*{Q. What software development environments are you most familiar with?}
My software development environment consists of two bash terminals on \gnu/Linux
operating system, side
by side.  In one terminal I write code with \texttt{vim} editor and in the other I
compile and execute. I mostly use the \gcc\ compilers and
\texttt{make}\footnote{\url{www.gnu.org/software/make}} or
\texttt{meson}\footnote{\url{mesonbuild.com}} for building.

\subsubsection*{Q. What software documentation tool are you most familiar with?} 
I hate Doxygen-produced documentation.  Though I consider software documentation
of utter importance, and I often start writting the manuals and such even before
the code itself. In the repositories I maintain, besides \emph{readme}s in
markdown I like to include tex manuscripts to produce human-made
documentation.

\subsection*{Programming Competency}
About a year ago I started working on a prototype for this \fft\ library.
The code is publicly available at \url{github.com/Lagrang3/fftx}.

